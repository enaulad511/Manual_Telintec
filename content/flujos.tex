\chapter{Principales Procesos}

Este módulo describe el flujo de validación de vales dentro del sistema Telintec, asegurando que cada entrega de productos esté correctamente registrada, aprobada y trazada. Es clave para mantener la integridad operativa entre almacén, administración y operaciones.

\subsection{Flujo de validación de vales}

El proceso de validación de vales sigue una secuencia estructurada que involucra distintos roles y etapas:

\begin{enumerate}
    \item \textbf{Creación del vale:} El usuario solicita un vale desde el módulo correspondiente, seleccionando productos, cantidades y contrato asociado.
    \item \textbf{Verificación de stock:} El sistema revisa la disponibilidad de los productos en el inventario.
    \item \textbf{Aprobación del vale:} El responsable del área revisa la solicitud y aprueba o rechaza según criterios operativos.
    \item \textbf{Despacho de productos:} Una vez aprobado, el almacén realiza la entrega física del producto.
    \item \textbf{Registro del movimiento:} El sistema actualiza el inventario y marca el vale como completado.
\end{enumerate}

\begin{figure}[h]
\centering
\begin{subfigure}{0.4\textwidth}
    \includegraphics[width=\textwidth]{imgs/no-image.png}
    \caption{Flujo de validación.}
    \label{fig:procesos1}
\end{subfigure}
\caption{Proceso completo de validación de vales en Telintec.}
\end{figure}

Este flujo garantiza trazabilidad, control y eficiencia en la entrega de materiales, alineando las acciones del sistema con los procesos internos de la empresa.
