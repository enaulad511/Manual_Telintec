
\chapter{Administración}

\begin{justify}
El módulo de \textbf{Administración} está diseñado para centralizar y facilitar la gestión operativa del sistema Telintec. Su propósito es brindar a los administradores herramientas clave para el control de contratos, compras, base de datos y solicitudes de material (SM), permitiendo una operación eficiente y trazable entre departamentos.

Este manual describe la estructura y funcionalidades actuales del módulo, y se actualizará conforme se integren nuevas capacidades. Las áreas que actualmente operan dentro de este módulo incluyen:
\end{justify}

\begin{itemize}
    \item \textbf{Gestión de Base de Datos}: Alta, baja, modificación y consulta de clientes y proveedores.
    \item \textbf{Contratos}: Creación, edición y actualización de contratos.
    \item \textbf{Órdenes de compra}: Generación y seguimiento de compras aprobadas.
    \item \textbf{Entrega de SM}: Administración de solicitudes de material utilizadas para solicitar productos al almacén.
\end{itemize}

\begin{justify}
Este módulo es fundamental para coordinar procesos entre administración, almacén y operaciones, asegurando que cada solicitud esté correctamente registrada, validada y trazada en el sistema.
\end{justify}

\newpage
\pagestyle{fancy}




\subsection{3.1 Base de datos}

\subsubsection{3.1.1 Gestión de Base de Datos – Pestaña: Clientes}

\begin{justify}
    La sección Gestión de Base de Datos está compuesta por dos pestañas principales: Clientes y Proveedores. En esta parte del manual se describe el funcionamiento de la pestaña Clientes, utilizada por el departamento de administración para registrar, consultar y mantener actualizada la información de las empresas o personas con las que se tiene relación comercial.
En esta pestaña puedes:
\end{justify}

En esta pestaña puedes:

\begin{itemize}
    \item Registrar nuevos clientes.
    \item Editar información existente.
    \item Buscar clientes por nombre o ID.
\end{itemize}

\begin{figure}[h]
\centering
\begin{subfigure}{0.8\textwidth}
    \includegraphics[width=\textwidth]{imgs/Administracion/BASE_DE_DATOS/administracion-BASEDEDATOS/ad_bd_tabla.png}
    \caption{Gestión de clientes.}
    \label{fig:admin1}
\end{subfigure}
\caption{Interfaz de clientes en la base de datos.}
\end{figure}


\subsection{Crear Cliente}

\begin{justify}
Esta sección está destinada al registro y gestión de la información de los clientes dentro del sistema Telintec. El formulario de creación permite ingresar los datos necesarios para dar de alta a un nuevo cliente.
\end{justify}

\subsection*{Campos disponibles para registro}

\begin{itemize}
    \item \textbf{C.U.I.T / RIF}: Identificador fiscal del cliente.
    \item \textbf{Nombre}: Nombre de la compañía o razón social.
    \item \textbf{Email}: Correo electrónico de contacto.
    \item \textbf{Teléfono}: Número telefónico del cliente.
    \item \textbf{RFC}: Registro Federal de Contribuyentes.
    \item \textbf{Dirección}: Ubicación física del cliente.
\end{itemize}

\subsection*{Acciones disponibles (Botones)}

\begin{itemize}
    \item \textbf{Agregar cliente}: Registra un nuevo cliente en el sistema.
    \item \textbf{Actualizar cliente}: Modifica los datos existentes de un cliente previamente registrado.
    \item \textbf{Eliminar cliente}: Elimina el registro de forma definitiva.  
    \textit{Esta acción no se puede revertir.}
    \item \textbf{Limpiar campos}: Restablece todos los campos del formulario para ingresar nuevos datos.
    \item \textbf{Mostrar tabla}: Visualiza el listado completo de clientes registrados en el sistema.
\end{itemize}
\subsection{Visualización de Clientes Registrados}

\begin{justify}
Dentro de la pestaña \textbf{Clientes}, el sistema permite consultar los registros existentes mediante el botón \textbf{Mostrar tabla}, ubicado en la parte inferior izquierda del formulario. Al hacer clic en este botón, se despliega una lista con todos los clientes registrados en el sistema, mostrando información clave para su identificación y gestión.
\end{justify}

\subsection*{Datos mostrados en la tabla}

\begin{itemize}
    \item \textbf{ID}
    \item \textbf{Nombre}
    \item \textbf{Email}
    \item \textbf{Teléfono}
    \item \textbf{RFC}
    \item \textbf{Dirección}
\end{itemize}

\subsection*{Edición directa desde la lista}

\begin{justify}
Cada elemento de la lista incluye un botón de \textbf{Editar}. Al presionarlo, el sistema carga automáticamente la información del cliente en el formulario superior.
\end{justify}

\begin{itemize}
    \item Permite modificar cualquier detalle del cliente sin buscarlo manualmente.
    \item Facilita la corrección de datos de forma inmediata.
    \item Optimiza la actualización de registros directamente desde la vista general.
\end{itemize}

\begin{justify}
Esta funcionalidad agiliza la gestión de registros, permitiendo una edición rápida y precisa directamente desde la tabla de clientes.
\end{justify}

\subsubsection{3.1.2 Gestión de Base de Datos – Pestaña: Proveedores}
\begin{justify}
    La pestaña Proveedores permite registrar, consultar y mantener actualizada la información de los proveedores que colaboran con la empresa. Esta funcionalidad es utilizada principalmente por el departamento de administración para asegurar que los datos comerciales estén disponibles y actualizados para procesos como compras, solicitudes de material (SM) y generación de órdenes.
\end{justify}



\begin{figure}[H]
\centering

\begin{subfigure}{0.45\textwidth}
    \centering
    \includegraphics[width=\textwidth]{imgs/Administracion/BASE_DE_DATOS/administracion-BASEDEDATOS/editar elemento de lista de provedores.png}
    \caption{Campos del proveedor}
    \label{fig:campos_proveedor}
\end{subfigure}
\hfill
\begin{subfigure}{0.45\textwidth}
    \centering
    \includegraphics[width=\textwidth]{imgs/Administracion/BASE_DE_DATOS/administracion-BASEDEDATOS/ad_bd_provedores_tabla.png}
    \caption{Tabla de proveedores}
    \label{fig:tabla_proveedor}
\end{subfigure}

\caption{Interfaz de gestión y visualización de proveedores.}
\label{fig:admin_proveedores}
\end{figure}



\subsubsection{Crear Proveedor}

\begin{justify}
Esta sección está destinada al registro y gestión de la información de los proveedores que colaboran con Telintec. El formulario permite ingresar datos fiscales, de contacto y comerciales para mantener una base de datos actualizada y confiable.
\end{justify}

\subsubsection*{Campos disponibles para registro}

\begin{itemize}
    \item \textbf{Proveedor}: Nombre comercial o razón social.
    \item \textbf{Nombre}: Persona de contacto principal.
    \item \textbf{Email}: Correo electrónico del proveedor (puede incluir múltiples contactos).
    \item \textbf{Teléfono}: Número telefónico del proveedor.
    \item \textbf{Dirección}: Ubicación física o fiscal del proveedor.
    \item \textbf{Marcas}: Marcas que representa o distribuye.
    \item \textbf{Tipo}: Clasificación del proveedor (por ejemplo: materiales, servicios).
    \item \textbf{RFC}: Registro Federal de Contribuyentes.
    \item \textbf{Sitio web}: Página oficial del proveedor.
\end{itemize}

\subsection*{Acciones disponibles (Botones)}

\begin{itemize}
    \item \textbf{Agregar proveedor}: Registra un nuevo proveedor en el sistema.
    \item \textbf{Actualizar proveedor}: Modifica los datos existentes de un proveedor previamente registrado.
    \item \textbf{Eliminar proveedor}: Elimina el registro de forma definitiva.  
    \textit{Esta acción no se puede revertir.}
    \item \textbf{Limpiar campos}: Restablece todos los campos del formulario para ingresar nuevos datos.
    \item \textbf{Mostrar tabla}: Visualiza el listado completo de proveedores registrados.
\end{itemize}

\subsubsection{Visualización de Proveedores Registrados}

\begin{justify}
Al hacer clic en el botón \textbf{Mostrar tabla}, se despliega una lista con todos los proveedores registrados. Esta tabla presenta información clave que permite una rápida identificación y acceso a los datos del proveedor.
\end{justify}

\subsection*{Datos mostrados en la tabla}

\begin{itemize}
    \item \textbf{Nombre del proveedor}
    \item \textbf{Persona de contacto}
    \item \textbf{Email}
    \item \textbf{Teléfono}
    \item \textbf{Sitio web}
\end{itemize}

\begin{justify}
Cada fila del listado permite seleccionar o editar la información del proveedor, facilitando una gestión rápida y eficiente desde la vista general.
\end{justify}

\subsection*{Edición directa desde la lista}

\begin{justify}
Cada elemento de la tabla incluye un botón de \textbf{Editar}. Al presionarlo, el sistema carga automáticamente la información del proveedor en el formulario superior.
\end{justify}

\begin{justify}
Esto permite modificar cualquier detalle del proveedor de forma rápida y precisa, sin necesidad de buscarlo manualmente dentro del listado.
\end{justify}

\subsection*{Acciones disponibles (Botones)}

\begin{itemize}
    \item \textbf{Agregar proveedor}: Registra un nuevo proveedor en el sistema.
    \item \textbf{Actualizar proveedor}: Modifica la información de un proveedor previamente registrado.
    \item \textbf{Eliminar proveedor}: Elimina el registro de forma definitiva.  
    \textit{Esta acción no se puede revertir.}
    \item \textbf{Limpiar campos}: Restablece todos los campos del formulario para ingresar nuevos datos.
    \item \textbf{Mostrar tabla}: Visualiza el listado completo de proveedores registrados.
\end{itemize}




\subsection{3.2 Contratos}

\begin{justify}
    
La sección \textbf{Contratos} forma parte del módulo de Administración y está diseñada para registrar, consultar y gestionar los contratos activos dentro de la plataforma Telintec. Esta funcionalidad permite llevar un control estructurado sobre los acuerdos operativos, facilitando el seguimiento de fechas, ubicaciones y demás información clave relacionada con cada contrato.
\end{justify}

\begin{figure}[H]
    \centering
    \includegraphics[width=0.85\textwidth]{imgs/Administracion/Contratos/contratos_1.png}
    \caption{Interfaz principal del módulo de Contratos.}
    \label{fig:administracion_contratos}
\end{figure}



\subsubsection{3.2.1 Crear Contrato}

\begin{justify}
    La funcionalidad Crear Contrato permite registrar nuevos acuerdos operativos en el sistema Telintec, asegurando que toda la información relevante quede documentada y disponible para consulta, seguimiento y validación. Esta acción se realiza desde la sección Contratos, ubicada dentro del módulo de Administración.

\end{justify}

\subsubsection*{Campos requeridos para el registro}

\begin{itemize}
    \item \textbf{ID del contrato}
    \item \textbf{Planta}
    \item \textbf{Área}
    \item \textbf{Nombre del contrato}
    \item \textbf{Actividades del contrato}
    \item \textbf{Descripción}
    \item \textbf{Observaciones}
    \item \textbf{Fecha de entrada / salida}
    \item \textbf{Fecha de contrato}
    \item \textbf{Fecha NOD}
    \item \textbf{Código de colocación}
    \item \textbf{Ubicación}
    \item \textbf{Identificador del contrato}
\end{itemize}

\subsubsection*{Funcionalidades: Sección Inferior}

\begin{itemize}
    \item \textbf{Desde documentos}
    \item \textbf{Cargar plantilla}: Permite adjuntar documentos relacionados al contrato (PDF, Word, imágenes, etc.).
\end{itemize}

\begin{figure}[H]
    \centering
    \includegraphics[width=0.85\textwidth]{imgs/Administracion/Contratos/contratos_d_documentos.png}
    \caption{Sección inferior para carga de documentos del contrato.}
    \label{fig:contratos_documentos}
\end{figure}

\begin{figure}[H]
    \centering
    \includegraphics[width=0.85\textwidth]{imgs/Administracion/Contratos/contratos_partidas_cargadas.png}
    \caption{Así se visualiza la carga de partidas.}
    \label{fig:contratos_partidas}
\end{figure}

\begin{justify}
\textbf{Advertencia importante:} Cada vez que se realice una carga de productos, si el archivo contiene errores o el usuario intenta cargar nuevamente, la información anterior será sobrescrita por la nueva.

Por lo tanto, asegúrate de revisar bien el archivo antes de subirlo.
\end{justify}

\subsubsection{Desde Datos}

\begin{itemize}
    \item \textbf{Ingreso manual del Producto:}  
    El usuario puede seleccionar el producto presionando el botón \textbf{“Agregar Producto”}.  
    Al hacerlo, se generará una ventana emergente que permitirá ingresar de forma manual el producto, registrándolo uno por uno.

    \item \textbf{Botón “Crear contrato”:}  
    Una vez completados todos los campos y agregados los productos correspondientes, este botón registra el contrato en el sistema.
\end{itemize}
\begin{figure}[H]
    \centering
    \includegraphics[width=0.85\textwidth]{imgs/Administracion/Contratos/contrato_desde_datos.png}
    \caption{Ventana emergente para agregar productos manualmente.}
    \label{fig:contratos_agregar_producto}
\end{figure}

\begin{justify}
Después de presionar el botón, inmediatamente se mostrará una ventana emergente que permitirá agregar un producto nuevo de forma manual. Esta interfaz está diseñada para capturar productos uno por uno, ingresando su información de manera precisa antes de guardarlo dentro del contrato.
\end{justify}
\begin{figure}[H]
    \centering
    \includegraphics[width=0.85\textwidth]{imgs/Administracion/Contratos/contratos_crear_new_product.png}
    \caption{Ventana emergente para agregar un nuevo producto.}
    \label{fig:contratos_ventana_agregar}
\end{figure}

\begin{justify}
Finalmente, el botón \textbf{“Crear contrato”} guardará toda la información capturada, incluyendo los datos generales del contrato y la lista de productos asociados.
\end{justify}

\begin{justify}
\textbf{Nota:} Es importante que todos los campos relevantes se llenen de forma correcta antes de crear o actualizar un contrato, ya que estos datos serán utilizados para reportes y seguimientos posteriores dentro del sistema.
\end{justify}

º

\subsubsection{Editar contrato}

\begin{justify}
La funcionalidad de \textbf{Editar contrato} permite modificar la información de un contrato previamente registrado en el sistema Telintec. Esta acción es esencial para mantener los datos actualizados y reflejar cualquier cambio en los términos, fechas o productos asociados al contrato.
\end{justify}

\begin{figure}[H]
    \centering
    \includegraphics[width=0.85\textwidth]{imgs/Administracion/Contratos/contratos_editar.png}
    \caption{Interfaz para editar un contrato existente.}
    \label{fig:editar_contrato}
\end{figure}
\begin{justify}
\textbf{Acceso a la edición:}  
Al seleccionar la pestaña \textbf{Editar}, se despliega una lista de contratos justo debajo del formulario principal. Cada contrato aparece en una fila con información clave como:
\end{justify}

\begin{itemize}
    \item \textbf{Abreviación}
    \item \textbf{Título}
    \item \textbf{Fecha de creación}
\end{itemize}

\subsection*{Botón de edición}

\begin{justify}
Cada fila incluye un botón con un ícono de \textit{pincel}, indicando que el contrato puede ser editado.
Al presionar este botón:
\end{justify}

\begin{itemize}
    \item El sistema auto-rellena los campos del formulario superior con la información del contrato seleccionado.
    \item Esto permite modificar directamente los datos sin necesidad de buscarlos manualmente.
\end{itemize}

\subsection*{Campos que se pueden editar}

\begin{justify}
Una vez cargado el contrato en el formulario, el usuario puede actualizar cualquier campo disponible, tales como:
\end{justify}

\begin{itemize}
    \item \textbf{Nombre del contrato}
    \item \textbf{Área}
    \item \textbf{Ubicación}
    \item \textbf{Actividades}
    \item \textbf{Fechas clave} (emisión, SICO, entrada/salida)
    \item \textbf{Entidad contratante / contratada}
    \item \textbf{Coordinador}
    \item \textbf{Observaciones}
\end{itemize}
\subsubsection{Botón de Actualizar Contrato}

\begin{justify}
El botón \textbf{Actualizar Contrato} permite guardar los cambios realizados en un contrato previamente cargado en el formulario, asegurando que toda la información modificada quede registrada correctamente en el sistema.
\end{justify}

\begin{justify}
Una vez que el usuario termina de editar los datos del contrato, únicamente deberá presionar este botón para confirmar y almacenar los cambios.
\end{justify}

\begin{figure}[H]
    \centering
    \includegraphics[width=0.65\textwidth]{imgs/Administracion/Contratos/contrato_btn_actualizar.png}
    \caption{Botón de actualización del contrato.}
    \label{fig:actualizar_contrato}
\end{figure}



\subsection{3.3 Órdenes de compra}

\subsubsection{3.3.1 Crear orden de compra desde solicitud de material (SM)}

Desde una SM aprobada:

\begin{itemize}
    \item Selecciona los productos requeridos.
    \item El sistema autocompleta los datos del proveedor y contrato.
    \item Genera la orden de compra con folio único.
\end{itemize}

\begin{justify}
    La sección Órdenes de Compra permite registrar, consultar y gestionar las compras realizadas por la empresa. Esta funcionalidad está integrada en el módulo de Administración y se vincula directamente con las solicitudes de material (SM) y los procesos de validación interna.

\end{justify}

\begin{figure}[h]
\centering
\begin{subfigure}{0.8\textwidth}
    \includegraphics[width=\textwidth]{imgs/Administracion/Ordenes_compra/od_comp_1.png}
    \caption{Orden desde SM.}
    \label{fig:admin4}
\end{subfigure}
\caption{Creación de orden de compra desde solicitud de material.}
\end{figure}

\subsubsection{Visualización de órdenes registradas}

En la parte superior de la sección se muestra una tabla con el listado de órdenes de compra existentes, incluyendo los siguientes campos:

\begin{itemize}
    \item \textbf{ID}: Número identificador de la orden.
    \item \textbf{Created by}: Usuario que generó la orden.
    \item \textbf{Estado}: Estado actual (por ejemplo, Pendiente).
    \item \textbf{Creación}: Fecha y hora de registro.
    \item \textbf{Acciones}: Botones disponibles para cada orden.
\end{itemize}

\textbf{Botones disponibles por orden:}
\begin{itemize}
    \item \textbf{Editar (ícono negro)}: Permite modificar los datos de la orden.
    \item \textbf{PDF (ícono verde)}: Descarga la orden en formato PDF para impresión o archivo.
    \item \textbf{Eliminar (ícono rojo)}: Elimina la orden del sistema. Esta acción es definitiva y no puede revertirse.
\end{itemize}

Esta sección es clave para el control de compras, permitiendo trazabilidad desde la solicitud hasta la ejecución, y facilitando la documentación formal de cada adquisición.

\subsubsection{Formas de crear una orden de compra}

El sistema ofrece dos métodos para generar una orden de compra:

\begin{enumerate}
    \item \textbf{Desde solicitud (SM)}  
    Utiliza la información previamente registrada en una solicitud de material.  
    Ideal para compras que ya fueron solicitadas y aprobadas por el área operativa.

    \item \textbf{Crear nueva orden de compra}  
    Permite registrar una orden desde cero, ingresando manualmente los datos requeridos.  
    Útil para compras directas o casos especiales que no provienen de una SM.
\end{enumerate}

Ambas opciones están disponibles mediante los botones:

\begin{itemize}
    \item \textbf{Crear O.C. desde Solicitud}
    \item \textbf{Crear Nueva O.C.}
\end{itemize}



\subsubsection{Crear orden de compra desde solicitud de material (SM)}

\textbf{Ventana Emergente: Crear Orden de Compra desde Solicitud}

En esta ventana se muestra una tabla con todas las solicitudes de material registradas en el sistema. Cada fila contiene información clave como:

\begin{itemize}
    \item \textbf{Referencia}: Código único de la solicitud.
    \item \textbf{Tipo}: Producto o servicio solicitado.
    \item \textbf{Creado por}: Usuario que generó la solicitud.
    \item \textbf{Estado}: Pendiente, Autorizado o Rechazado.
    \item \textbf{Fecha de creación}
    \item \textbf{Acciones}: Botones disponibles para cada solicitud.
\end{itemize}

\begin{figure}[H]
    \centering
    \includegraphics[width=12cm]{imgs/Administracion/Ordenes_compra/od_comp_desde_solicitud.png}
    \caption{Ventana de creación de Orden desde Solicitud SM.}
\end{figure}

\subsubsection{Botón de edición}

Cada solicitud incluye un botón con ícono de lápiz, que permite editar o seleccionar la solicitud para convertirla en una orden de compra. 

Al presionar este botón:

\begin{itemize}
    \item El sistema auto-rellena los campos del formulario que se mostrará en la ventana emergente con la información de la solicitud seleccionada.
    \item Esto incluye productos, cantidades, proveedor sugerido y observaciones relevantes.
    \item El usuario puede revisar y ajustar los datos antes de confirmar la creación de la orden.
\end{itemize}

Esta funcionalidad agiliza el proceso de compras al reutilizar información ya validada por el área operativa, asegurando coherencia entre lo solicitado y lo adquirido.

\subsubsection{Ventana Emergente: Crear Orden de Compra}

Al seleccionar una solicitud desde la ventana emergente de ``Crear O.C. desde solicitud'', el sistema carga automáticamente los datos de la solicitud de material y los relaciona en el formulario de la orden de compra.

\textbf{Auto-relleno de datos}

Al presionar el botón de edición sobre una solicitud de material (SM), el sistema auto-rellena los campos del formulario con:

\begin{itemize}
    \item Productos solicitados
    \item Cantidades
    \item Proveedor sugerido
    \item Observaciones
    \item Sucursal y folio de la SM
\end{itemize}

\begin{figure}[H]
    \centering
    \includegraphics[width=12cm]{imgs/Administracion/Ordenes_compra/od_comp_desde_soli_table_2.png}
    \caption{Vista del botón de edición y auto-relleno de datos en el formulario}
\end{figure}

\subsubsection{Modes de creación disponibles}

El sistema ofrece tres modos de creación de orden de compra, según el origen o tipo de operación:

\begin{enumerate}
    \item \textbf{General} \\
    Para compras institucionales o sin un solicitante específico.

    \begin{figure}[H]
        \centering
        \includegraphics[width=12cm]{imgs/Administracion/Ordenes_compra/od_comp_crear_1.png}
        \caption{Modo General de creación de orden de compra}
    \end{figure}

    \item \textbf{Solicitante} \\
    Asociadas directamente a una persona o área que realizó la solicitud.

    \begin{figure}[H]
        \centering
        \includegraphics[width=12cm]{imgs/Administracion/Ordenes_compra/od_comp_crear_solicitante.png}
        \caption{Modo Solicitante de creación de orden de compra}
    \end{figure}

    \item \textbf{Proveedor} \\
    Enfocadas en compras directas a un proveedor específico, incluso si no hay SM previa.

    \begin{figure}[H]
        \centering
        \includegraphics[width=12cm]{imgs/Administracion/Ordenes_compra/od_comp_crear_proveedor.png}
        \caption{Modo Proveedor de creación de orden de compra}
    \end{figure}

\end{enumerate}

\subsubsection{Gestión de productos (Items)}

En cualquiera de los tres modos, el usuario puede administrar los productos incluidos en la orden:

\begin{itemize}
    \item \textbf{Agregar Item (Botón)}:  
    Permite añadir un nuevo producto al listado, ingresando descripción, cantidad, marca, categoría, número de parte, proveedor y precio unitario.

    \item \textbf{Eliminar Item}:  
    Si un producto no es necesario o fue agregado por error, puede eliminarse mediante el botón rojo \textbf{Eliminar} junto al producto correspondiente.
\end{itemize}

\textbf{Finalizar orden}

Una vez revisados y completados todos los campos, el usuario debe presionar el botón \textbf{``Crear Orden''} para registrar la orden de compra en el sistema.


\subsubsection{3.3.2 Crear Nueva orden de compra (ingresando manualmente los datos)}

\begin{justify}
La opción \textbf{Crear Nueva O.C.} permite registrar una orden de compra desde cero, sin depender de una solicitud de material previa. Esta funcionalidad es útil para compras directas, adquisiciones especiales o casos donde se requiere mayor flexibilidad en el ingreso de datos.
\end{justify}

\begin{figure}[H]
    \centering
    \includegraphics[width=12cm]{imgs/Administracion/Ordenes_compra/od_comp_crear_manualmet.png}
    \caption{Vista del formulario para crear una nueva orden de compra}
\end{figure}

\subsubsection{Formulario de registro}

\begin{justify}
Al presionar el botón \textbf{``Crear Nueva O.C.''}, se despliega un formulario completo dividido en secciones:
\end{justify}

\begin{enumerate}
    \item \textbf{General}: Para compras institucionales o sin un solicitante específico.
    \item \textbf{Solicitante}: Asociadas directamente a una persona o área que realizó la solicitud.
    \item \textbf{Proveedor}: Enfocadas en compras directas a un proveedor específico, incluso si no hay SM previa.
\end{enumerate}

\subsubsection*{Sección de productos (Items)}

\begin{justify}
En esta sección se ingresan manualmente los productos o servicios que serán incluidos en la orden:
\end{justify}

\begin{itemize}
    \item Descripción
    \item Cantidad
    \item Marca
    \item Categoría
    \item Número de parte
    \item Proveedor
    \item Unidad
    \item Duración de servicios (si aplica)
    \item Precio unitario
    \item Producto en inventario: Indica si el producto está registrado en el sistema o es externo.
\end{itemize}

\textbf{Funciones disponibles:}
\begin{itemize}
    \item \textbf{Agregar Item}: Permite añadir otro producto al listado.
    \item \textbf{Eliminar Item}: Si un producto fue ingresado por error, puede eliminarse con el botón rojo \textbf{Eliminar}.
\end{itemize}

\subsubsection*{Finalizar orden}

\begin{justify}
Una vez completados todos los campos y revisados los productos, el usuario debe presionar el botón \textbf{``Crear Orden''} para registrar la orden en el sistema.
\end{justify}

\subsubsection{Solicitante}

\begin{justify}
Persona o área que solicita la compra. Este modo permite asociar la orden de compra directamente con quien hizo la solicitud, facilitando el control y rastreo interno.
\end{justify}

\begin{figure}[H]
    \centering
    \includegraphics[width=12cm]{imgs/Administracion/Ordenes_compra/od_comp_crear_manualmente_solicitante.png}
    \caption{Vista del modo Solicitante en el formulario}
\end{figure}
\subsubsection{Proveedor}

\begin{justify}
Enfocado en compras directas a un proveedor específico, incluso si no existe una solicitud de material previa. Ideal para compras urgentes o acuerdos comerciales directos.
\end{justify}

\begin{figure}[H]
    \centering
    \includegraphics[width=12cm]{imgs/Administracion/Ordenes_compra/od_comp_crear_manual_proveedor.png}
    \caption{Vista del modo Proveedor en el formulario}
\end{figure}





\subsection{3.4 Entregas de Solicitudes de material (SM)}
\begin{justify}
La sección \textbf{Entregas de SM} permite registrar y gestionar la entrega de materiales solicitados por diferentes áreas de la empresa. Esta funcionalidad forma parte del módulo de Administración y está diseñada para asegurar que cada solicitud sea atendida correctamente, con trazabilidad y control sobre lo entregado.
\end{justify}


Para entregar materiales:

\begin{itemize}
    \item Selecciona una SM pendiente.
    \item Verifica productos disponibles y faltantes.
    \item Despacha los productos desde el almacén.
\end{itemize}


\begin{figure}[h]
\centering
\begin{subfigure}{0.8\textwidth}
    \includegraphics[width=\textwidth]{imgs/Administracion/Entregas_sm/entregas_sm_1.png}
    \caption{Entrega de SM.}
    \label{fig:admin5}
\end{subfigure}
\caption{Interfaz de entrega de solicitudes de material.}
\end{figure}


\subsubsection{3.4.1 Paso 1: Seleccionar una solicitud de material (SM)}
\begin{justify}
En la parte superior derecha de la pantalla encontrarás un selector desplegable. Desde ahí puedes elegir una solicitud de material registrada previamente. Al seleccionar una SM, el sistema carga automáticamente toda la información relacionada con esa solicitud. Los campos del formulario se rellenan de forma automática y no son editables, ya que reflejan los datos originales de la solicitud. Una vez cargada, el sistema muestra una notificación confirmando que la SM ha sido cargada correctamente.
\end{justify}

\textbf{Información mostrada en el formulario}

\begin{itemize}
    \item Fecha de solicitud
    \item Contrato asociado
    \item Usuario solicitante
    \item Personal técnico
    \item Número de pedido
    \item Planta
    \item Área / Ubicación
    \item Fecha crítica de entrega
    \item Área dirigida
\end{itemize}

\begin{justify}
Estos datos sirven como referencia para validar que estás trabajando sobre la solicitud correcta.
\end{justify}
\subsection{Items y Entregas}

\begin{justify}
La tabla de \textbf{Bienes solicitados y entregados}, ubicada en la parte inferior del formulario, es donde se gestiona la entrega real de los materiales solicitados. Aunque el formulario principal se rellena automáticamente y no es editable, esta tabla sí permite interacción directa por parte del usuario.
\end{justify}

\textbf{Campos visibles en la tabla:}

\begin{itemize}
    \item \textbf{Descripción}: Nombre del material solicitado.
    \item \textbf{Cantidad}: Total solicitado en la SM.
    \item \textbf{Unidad}: Unidad de medida (ej. pieza, metro, litro).
    \item \textbf{Entregado}: Cantidad ya entregada.
    \item \textbf{Entregar}: Cantidad que se desea entregar en este momento.
    \item \textbf{Estado}: Indica si el material ya fue entregado o está pendiente.
    \item \textbf{Acciones}: Opciones disponibles para cada ítem.
\end{itemize}

\begin{figure}[H]
    \centering
    \includegraphics[width=12cm]{imgs/Administracion/Entregas_sm/entregas_sm_guardar_entregas.png}
    \caption{Vista de la tabla de Items y su gestión de entregas}
\end{figure}

\subsubsection*{Acciones (Botones)}

\begin{justify}
En la columna \textbf{Acciones}, el usuario puede registrar la entrega de cada ítem:
\end{justify}

\begin{itemize}
    \item Si el material está disponible, se puede indicar como \textbf{Stock suficiente}.
    \item Si no hay disponibilidad, se puede marcar como \textbf{Stock insuficiente}.
    \item Al ingresar una cantidad en el campo \textbf{Entregar}, el sistema actualiza automáticamente el estado del ítem.
\end{itemize}

\begin{justify}
Esto permite llevar un control preciso de lo que se entrega en cada sesión, sin alterar la solicitud original.
\end{justify}

\subsubsection*{Guardar entregas}

\begin{justify}
Una vez que se han registrado las cantidades entregadas para cada ítem:
\end{justify}

\begin{itemize}
    \item Verifica que los datos estén correctos.
    \item Presiona el botón \textbf{``Guardar entregas''} para confirmar y registrar la operación en el sistema.
\end{itemize}

\begin{justify}
Esto asegura que la entrega quede documentada y que el área solicitante pueda continuar con el proceso de validación o firma.
\end{justify}
