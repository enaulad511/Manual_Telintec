




\section{Escaner}

La pestaña del escáner está enfocada en optimizar el registro de movimientos de inventario mediante el uso de tecnología de códigos de barras. Para la aplicación de escritorio se despliega una ventana extra donde se realizan las operaciones del lector. Las funciones disponibles son: 

\begin{figure}[ht!]
\centering
\includegraphics[width=0.7\textwidth]{imgs/LectorApp.png}
\caption{Ventana emergente para funcines del lector.}
\label{fig:lecotr}
\end{figure}

Permite a los usuarios registrar entradas y salidas de manera rápida y precisa escaneando los códigos de barras de los productos. Esto facilita la gestión del inventario de manera eficiente y sin errores manuales.

\textbf{¿El escáner se desconfiguró?} Sigue estos pasos:
\begin{enumerate}
    \item Verifica la conexión: Asegúrate de que el escáner esté correctamente conectado al dispositivo. Si es inalámbrico, comprueba que esté dentro del rango de la red o que la batería esté cargada. 

    \item Reinicia el escáner: Apaga el escáner y vuélvelo a encender para restablecer su configuración. 

    \item Reconfigura el escáner: 
    \begin{itemize}
        \item Escanea el código de configuración que aparece en el manual del escáner para restablecer los ajustes predeterminados.
        \item Puedes acceder al manual desde el 
        \href{https://drive.google.com/file/d/1OsTJD1-Hbvdd9wIGknSJqKVfENMQDajR/view?usp=sharing.}{\emph{enlace}}. 
        \item Este manual incluye los códigos necesarios y las instrucciones detalladas para la configuración.
    \end{itemize}
    \item Verifica la configuración del software. Asegúrate de buscar en tu dispositivo el administrador de dispositivos los puertos (COM) y poder ver si el escáner ya es reconocido por el dispositivo.

    \item Consulta la guía de solución de problemas. Si el escáner sigue sin funcionar correctamente, consulta la guía de solución de problemas para verificar posibles errores de lectura o ajustes necesarios. 
\end{enumerate}