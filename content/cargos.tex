\chapter{Cargos}

Este módulo permite gestionar los cargos asignados a líderes, gerentes y otros colaboradores dentro de la organización. Su objetivo es mantener una estructura clara de responsabilidades y facilitar la trazabilidad de acciones en el sistema.

\subsection{Asignación de cargos}

La asignación de cargos se realiza desde una interfaz que permite:

\begin{itemize}
    \item Registrar nuevos cargos para usuarios existentes.
    \item Visualizar la lista de líderes y gerentes con sus cargos actuales.
    \item Editar cargos asignados previamente.
\end{itemize}

\begin{figure}[h]
\centering
\begin{subfigure}{0.4\textwidth}
    \includegraphics[width=\textwidth]{imgs/no-image.png}
    \caption{Asignación de cargos.}
    \label{fig:cargos1}
\end{subfigure}
\caption{Interfaz para asignar cargos a colaboradores.}
\end{figure}

\subsection{Lista de líderes y gerentes con sus cargos}

La pantalla muestra una tabla con los siguientes campos:

\begin{itemize}
    \item Nombre del colaborador.
    \item Cargo asignado.
    \item Departamento.
    \item Fecha de asignación.
\end{itemize}

Esta lista permite filtrar por departamento o buscar por nombre para facilitar la gestión.

\begin{figure}[h]
\centering
\begin{subfigure}{0.4\textwidth}
    \includegraphics[width=\textwidth]{imgs/no-image.png}
    \caption{Lista de cargos.}
    \label{fig:cargos2}
\end{subfigure}
\caption{Visualización de líderes y gerentes con sus cargos.}
\end{figure}

\subsection{Editar cargo}

Para modificar un cargo:

\begin{itemize}
    \item Selecciona al colaborador desde la lista.
    \item Haz clic en el botón “Editar”.
    \item Actualiza el cargo y guarda los cambios.
\end{itemize}

El sistema registrará la modificación y actualizará la trazabilidad del usuario.

\begin{figure}[h]
\centering
\begin{subfigure}{0.4\textwidth}
    \includegraphics[width=\textwidth]{imgs/no-image.png}
    \caption{Edición de cargo.}
    \label{fig:cargos3}
\end{subfigure}
\caption{Formulario para editar cargos asignados.}
\end{figure}