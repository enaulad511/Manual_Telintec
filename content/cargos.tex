\chapter{Cargos}

\begin{justify}
El módulo \textbf{Cargos} está diseñado para gestionar la asignación de puestos dentro de la organización, de forma estructurada y controlada según los permisos del usuario. Esta funcionalidad permite visualizar, editar y administrar los roles que ocupan los empleados en cada departamento, asegurando que la jerarquía y responsabilidades estén correctamente documentadas.
\end{justify}

\begin{justify}
La visibilidad y capacidad de edición dentro del módulo varía según el área a la que pertenece el usuario:
\end{justify}

\begin{itemize}
    \item \textbf{Recursos Humanos (RH)}  
    \begin{itemize}
        \item Puede agregar, editar y eliminar puestos de todo el personal.
        \item Tiene acceso completo a la estructura organizacional.
    \end{itemize}

    \item \textbf{Dirección}  
    \begin{itemize}
        \item Puede gestionar exclusivamente los puestos asignados al área directiva.
        \item Ideal para validar cargos estratégicos y de liderazgo.
    \end{itemize}

    \item \textbf{Operaciones}  
    \begin{itemize}
        \item Accede a los puestos relacionados con el área operativa, como líderes de campo, supervisores y técnicos.
        \item Permite mantener actualizada la estructura funcional del equipo operativo.
    \end{itemize}

    \item \textbf{Administración}  
    \begin{itemize}
        \item Dirigido a jefes y gerentes de departamento.
        \item Permite revisar y ajustar los cargos internos dentro de su área específica.
    \end{itemize}
\end{itemize}

\begin{justify}
Este módulo es clave para mantener la trazabilidad de responsabilidades, facilitar procesos de validación y asegurar que cada empleado esté correctamente vinculado a su rol dentro del sistema.
\end{justify}



\newpage
\pagestyle{fancy}



\section{Asignación de cargos}
\begin{justify}
La sección \textbf{Asignación de Cargos} permite visualizar y administrar los puestos asignados a cada empleado dentro de la organización. Esta funcionalidad es clave para mantener actualizada la estructura operativa y jerárquica de cada departamento.
\end{justify}

\begin{justify}
    La asignación de cargos se realiza desde una interfaz que permite:
\end{justify}

\begin{itemize}
    \item Registrar nuevos cargos para usuarios existentes.
    \item Visualizar la lista de líderes y gerentes con sus cargos actuales.
    \item Editar cargos asignados previamente.
\end{itemize}


\begin{figure}[h]
\centering
\begin{subfigure}{0.8\textwidth}
    \includegraphics[width=\textwidth]{imgs/Cargos/cargos_1.png}
    \caption{Asignación de cargos.}
    \label{fig:cargos1}
\end{subfigure}
\caption{Interfaz para asignar cargos a colaboradores.}
\end{figure}




\subsection{Lista de líderes y gerentes con sus cargos}
\begin{justify}
    La sección \textbf{Lista de líderes y gerentes con sus cargos} ofrece una visión detallada de la estructura organizacional, mostrando quiénes ocupan los puestos clave dentro de la empresa. Esta funcionalidad facilita la gestión y supervisión de los roles asignados, asegurando que la información esté siempre actualizada y accesible para los usuarios autorizados.
    En esta pantalla se presenta una tabla que muestra los cargos asignados dentro de la organización, permitiendo una visualización clara de la estructura jerárquica y operativa. La tabla está organizada por los siguientes campos:
\end{justify}

\begin{itemize}
    \item \textbf{Número}: Identificador único de la asignación.
    \item \textbf{Nombre}: Título del cargo (por ejemplo: Gerente Operaciones, Líder CCTV).
    \item \textbf{Empleado}: Nombre completo del colaborador asignado.
    \item \textbf{Departamento}: Área a la que pertenece el cargo (por ejemplo: Operaciones, Dirección).
    \item \textbf{Cargos}: Subcategoría o proyecto específico asociado (por ejemplo: SITE, CCTV, RFID).
    \item \textbf{Acciones}: Íconos disponibles para editar o eliminar la asignación.
\end{itemize}
La pantalla muestra una tabla con los siguientes campos:
\begin{itemize}
    \item Nombre del colaborador.
    \item Cargo asignado.
    \item Departamento.
    \item Fecha de asignación.
\end{itemize}
\subsubsection{Selector de Departamento}

\begin{justify}
    En la parte derecha de la pantalla, el sistema incluye un selector desplegable que permite al usuario elegir el departamento del cual desea consultar o editar los cargos asignados. Este selector filtra la tabla de asignaciones, mostrando únicamente los cargos correspondientes al departamento seleccionado. Esto resulta especialmente útil para usuarios que tienen acceso a múltiples áreas o que necesitan gestionar roles específicos dentro de su unidad.
    
\end{justify}

\subsubsection*{Acceso según rol}

\begin{itemize}
    \item Si el usuario es líder o gerente de un departamento, podrá editar los cargos dentro del área seleccionada.
    \item Si el usuario no tiene permisos sobre ese departamento, el sistema bloqueará las acciones de edición, permitiendo solo la visualización.
\end{itemize}

\begin{figure}[h]
\centering
\begin{subfigure}{0.8\textwidth}
    \includegraphics[width=\textwidth]{imgs/Cargos/cargos_selector.png}
    \caption{Selector de Departamento.}
    \label{fig:cargos2}
\end{subfigure}
\caption{Selector de Departamento.}
\end{figure}



\subsection{Editar cargo}
\begin{justify}
    Al presionar el ícono de editar, se despliega una ventana emergente que permite modificar la asignación del cargo. Esta ventana incluye:
\end{justify}

\begin{itemize}
    \item \textbf{Titular:} Selección del empleado principal que ocupará el cargo.
    \item \textbf{Otros:} Campo para agregar colaboradores adicionales vinculados al mismo cargo.
    \item \textbf{Botón ``Guardar'':} Registra los cambios realizados.
\end{itemize}
\begin{justify}
    Esta funcionalidad está protegida por permisos y solo se habilita si el usuario tiene autoridad sobre el área correspondiente.
    Para modificar un cargo:
\end{justify}
\begin{itemize}
    \item Selecciona al colaborador desde la lista.
    \item Haz clic en el botón “Editar”.
    \item Actualiza el cargo y guarda los cambios.
\end{itemize}
\begin{justify}
    El sistema registrará la modificación y actualizará la trazabilidad del usuario.
\end{justify}

\begin{figure}[h]
\centering
\begin{subfigure}{0.4\textwidth}
    \includegraphics[width=\textwidth]{imgs/Cargos/cargo_editar.png}
    \caption{Edición de cargo.}
    \label{fig:cargos3}
\end{subfigure}
\caption{Formulario para editar cargos asignados.}
\end{figure}