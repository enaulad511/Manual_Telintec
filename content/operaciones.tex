\chapter{Operaciones}

Este módulo permite gestionar los vales de seguridad y herramientas/equipo, facilitando su creación, edición y trazabilidad. Está diseñado para que los responsables de almacén y operaciones puedan administrar entregas de forma eficiente.

\subsection{Vales}

\subsubsection{Ventana General de Vales}

La ventana principal muestra todos los vales registrados en el sistema, con filtros por tipo, estado y fecha.

\begin{itemize}
    \item Visualización de vales activos, completados o pendientes.
    \item Filtros por tipo de vale: seguridad o herramientas/equipo.
    \item Acciones disponibles: editar, imprimir, exportar.
\end{itemize}

\begin{figure}[h]
\centering
\begin{subfigure}{0.4\textwidth}
    \includegraphics[width=\textwidth]{imgs/no-image.png}
    \caption{Vista general de vales.}
    \label{fig:operaciones1}
\end{subfigure}
\caption{Panel principal del módulo de vales.}
\end{figure}

\subsubsection{Crear vale de seguridad}

Para crear un vale de seguridad:

\begin{itemize}
    \item Selecciona el colaborador y el contrato asociado.
    \item Añade los productos de seguridad requeridos desde el catálogo.
    \item Verifica el stock disponible.
    \item Guarda el vale para que sea procesado por el área correspondiente.
\end{itemize}

\begin{figure}[h]
\centering
\begin{subfigure}{0.4\textwidth}
    \includegraphics[width=\textwidth]{imgs/no-image.png}
    \caption{Creación de vale de seguridad.}
    \label{fig:operaciones2}
\end{subfigure}
\caption{Formulario para registrar un vale de seguridad.}
\end{figure}

\subsubsection{Editar vale de seguridad}

Para modificar un vale existente:

\begin{itemize}
    \item Localiza el vale en la lista general.
    \item Haz clic en “Editar”.
    \item Actualiza productos, cantidades o datos del colaborador.
    \item Guarda los cambios para actualizar el registro.
\end{itemize}

\begin{figure}[h]
\centering
\begin{subfigure}{0.4\textwidth}
    \includegraphics[width=\textwidth]{imgs/no-image.png}
    \caption{Edición de vale de seguridad.}
    \label{fig:operaciones3}
\end{subfigure}
\caption{Interfaz para editar vales de seguridad.}
\end{figure}

\subsubsection{Crear vale de Herramientas/Equipo}

Este tipo de vale permite registrar la entrega de herramientas o equipos específicos.

\begin{itemize}
    \item Selecciona el tipo de herramienta o equipo.
    \item Asocia el vale a un contrato y colaborador.
    \item Verifica disponibilidad y guarda el registro.
\end{itemize}

\begin{figure}[h]
\centering
\begin{subfigure}{0.4\textwidth}
    \includegraphics[width=\textwidth]{imgs/no-image.png}
    \caption{Vale de herramientas/equipo.}
    \label{fig:operaciones4}
\end{subfigure}
\caption{Formulario para crear vales de herramientas.}
\end{figure}

\subsubsection{Editar vale de Herramientas/Equipo}

Para modificar un vale de herramientas:s

\begin{itemize}
    \item Accede a la lista de vales.
    \item Selecciona el vale a editar.
    \item Cambia productos, cantidades o datos del responsable.
    \item Guarda los cambios.
\end{itemize}

\begin{figure}[h]
\centering
\begin{subfigure}{0.4\textwidth}
    \includegraphics[width=\textwidth]{imgs/no-image.png}
    \caption{Edición de vale de herramientas.}
    \label{fig:operaciones5}
\end{subfigure}
\caption{Interfaz para editar vales de herramientas/equipo.}
\end{figure}