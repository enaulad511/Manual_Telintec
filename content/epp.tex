\chapter{Almacén de Productos de Seguridad (EPP)}

Este módulo está diseñado para gestionar el inventario y la aprobación de vales relacionados con equipos de protección personal (EPP). Su objetivo es asegurar que los productos de seguridad estén disponibles, correctamente registrados y entregados conforme a los protocolos establecidos.

\subsection{Inventario de productos de Seguridad (EPP)}

En esta sección, los usuarios pueden consultar y administrar los productos de seguridad disponibles en el almacén.

\begin{itemize}
    \item Visualización de productos por categoría (guantes, cascos, chalecos, etc.).
    \item Registro de nuevos productos de seguridad.
    \item Edición de información como marca, proveedor y stock.
    \item Filtros por tipo de producto y proveedor.
\end{itemize}

\begin{figure}[h]
\centering
\begin{subfigure}{0.4\textwidth}
    \includegraphics[width=\textwidth]{imgs/no-image.png}
    \caption{Inventario de EPP.}
    \label{fig:epp1}
\end{subfigure}
\caption{Vista del inventario de productos de seguridad.}
\end{figure}

\subsection{Aprobar Vales de Seguridad}

Esta función permite validar y autorizar la entrega de productos de seguridad solicitados por los colaboradores.

\begin{itemize}
    \item Revisión de solicitudes pendientes.
    \item Verificación de disponibilidad en inventario.
    \item Aprobación o rechazo de vales según criterios operativos.
    \item Registro automático del movimiento en el sistema.
\end{itemize}

\begin{figure}[h]
\centering
\begin{subfigure}{0.4\textwidth}
    \includegraphics[width=\textwidth]{imgs/no-image.png}
    \caption{Aprobación de vales.}
    \label{fig:epp2}
\end{subfigure}
\caption{Interfaz para aprobar vales de seguridad.}
\end{figure}

\subsection{Flujo operativo de aprobación de vales}

El proceso de aprobación sigue un flujo estructurado:

\begin{enumerate}
    \item El usuario solicita un vale de seguridad desde el módulo correspondiente.
    \item El sistema verifica el stock disponible.
    \item El responsable revisa la solicitud y aprueba o rechaza.
    \item Si se aprueba, el producto se despacha y se registra el movimiento.
    \item El vale queda marcado como completado en el sistema.
\end{enumerate}

\begin{figure}[h]
\centering
\begin{subfigure}{0.4\textwidth}
    \includegraphics[width=\textwidth]{imgs/no-image.png}
    \caption{Flujo de aprobación.}
    \label{fig:epp3}
\end{subfigure}
\caption{Proceso operativo para vales de seguridad.}
\end{figure}